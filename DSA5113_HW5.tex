\documentclass[12pt, letterpaper]{article}
\usepackage[utf8]{inputenc}
\usepackage{titlesec}
\usepackage{graphicx}
\usepackage{mathtools}		% For piecewise functions



\titleformat*{\section}{\normalsize\bfseries}
\titleformat*{\subsection}{\small\bfseries}
\titleformat*{\subsubsection}{\small\bfseries}
\titleformat*{\paragraph}{\large\bfseries}
\titleformat*{\subparagraph}{\large\bfseries}

\begin{document}
\begin{center}
\Large DSA/ISE 5113 Advanced Analytics and Metaheuristics
\Large Homework \#5\\
\vspace{3mm}
\normalsize April 19, 2020\\
\vspace{3mm}
\normalsize Rachel Bennett
\end{center}


\section*{Question 1: Simulated Annealing}
\begin{itemize}
\item The starting temperature was found by using the current best solution value found (15218.80, from Random Restart). This was compared with the initial temperature, which in this case had a very negative value due to penalization of solutions over the weigh limit.The initial temperature was adjusted until a 0.99 probability of acceptance was found( in this case, $T_0 =10,000,000,00$).
\item The temperature cooling schedule was fairly basic, with each round of cooling updating like so:
$$ T = \alpha * T$$
Each temperature was held for a static 10 iterations before being cooled to a lower one.
\item The search proceeded until the temperature was below 1. 
\end{itemize}

\section*{Question 2: Variable Neighborhood Search}
Variable Neighborhood Search had far and away the best results, with a final value of 19310.70. The neighborhoods used were the 1, 2, and 3 flip neighborhoods, and the local search procedure implemented was Variable Neighborhood Descent. In order for the program to reach a solution in a reasonable amount of time, the local search was limited to run for only 1 minute per search. 

\section*{Question 3: Tabu Search}

Tabu search was implemented with a tabu criterion of 'if an item was in the previous solution, it is now tabu'. The tenure for this was 5 iterations, and the aspiration criterion was any solution better than the current best. Long term memory was implemented to keep track of each item's residence measure, and a solution was considered more appealing by how many items in it's solution had a high residence. Stopping criterion was a simplistic limit of 5000 iterations.

\section*{Question 4: Guided Local Search}
Guided local search was set with a dynamic lambda that updated with a given neighborhood's minimum, and tuned with an alpha of 1. The stopping criterion was 10000 iterations. This search procedure ran very well, giving a final objective value of 17784.6

\begin{center}
\begin{tabular}{ l r r r r } 

 \multicolumn{5}{c}{Results} \\
 \hline
Algorithm & Iterations & \# Items Selected & Weight & Objective \\
 \hline 
Local Search (Best Improvement) & 7050 & 28 & 1499.6 & 14170.60 \\
Local Search (First Improvement) & 3055 & 16 & 1497.1 & 6391.79\\
Local Search (Random Restarts) & 71400 & 31 & 1494.6 & 15218.80\\
Local Search (Random Walk) & 6635 & 19 & 1495.1 & 7801.60\\
Local Beam Search & 12600 & 26 & 1499.3 & 11866.50\\
Simulated Annealing & 309300 & 25 & 1495.1 & 14766.40\\
Variable Neighborhood Search & 307280 & 31 & 1494.4 &  19310.70\\
Tabu Search & 5000 & 27 & 1490.4 & 14184.10\\
Guided Local Search &  10050 & 31 & 1493.2 & 17784.60\\
 \hline
\end{tabular}
\end{center}

\end{document}